%==============================================================================
% tento soubor pouzijte jako zaklad
% this file should be used as a base for the thesis
% Autoři / Authors: 2008 Michal Bidlo, 2019 Jaroslav Dytrych
% Kontakt pro dotazy a připomínky: sablona@fit.vutbr.cz
% Contact for questions and comments: sablona@fit.vutbr.cz
%==============================================================================
% kodovani: UTF-8 (zmena prikazem iconv, recode nebo cstocs)
% encoding: UTF-8 (you can change it by command iconv, recode or cstocs)
%------------------------------------------------------------------------------
% zpracování / processing: make, make pdf, make clean
%==============================================================================
% Soubory, které je nutné upravit nebo smazat: / Files which have to be edited or deleted:
%   projekt-20-literatura-bibliography.bib - literatura / bibliography
%   projekt-01-kapitoly-chapters.tex - obsah práce / the thesis content
%   projekt-01-kapitoly-chapters-en.tex - obsah práce v angličtině / the thesis content in English
%   projekt-30-prilohy-appendices.tex - přílohy / appendices
%   projekt-30-prilohy-appendices-en.tex - přílohy v angličtině / appendices in English
%==============================================================================
\documentclass[]{fitthesis} % bez zadání - pro začátek práce, aby nebyl problém s překladem
%\documentclass[english]{fitthesis} % without assignment - for the work start to avoid compilation problem
%\documentclass[zadani]{fitthesis} % odevzdani do wisu a/nebo tisk s barevnými odkazy - odkazy jsou barevné
%\documentclass[english,zadani]{fitthesis} % for submission to the IS FIT and/or print with color links - links are color
%\documentclass[zadani,print]{fitthesis} % pro černobílý tisk - odkazy jsou černé
%\documentclass[english,zadani,print]{fitthesis} % for the black and white print - links are black
%\documentclass[zadani,cprint]{fitthesis} % pro barevný tisk - odkazy jsou černé, znak VUT barevný
%\documentclass[english,zadani,cprint]{fitthesis} % for the print - links are black, logo is color
% * Je-li práce psaná v anglickém jazyce, je zapotřebí u třídy použít 
%   parametr english následovně:
%   If thesis is written in English, it is necessary to use 
%   parameter english as follows:
%      \documentclass[english]{fitthesis}
% * Je-li práce psaná ve slovenském jazyce, je zapotřebí u třídy použít 
%   parametr slovak následovně:
%   If the work is written in the Slovak language, it is necessary 
%   to use parameter slovak as follows:
%      \documentclass[slovak]{fitthesis}
% * Je-li práce psaná v anglickém jazyce se slovenským abstraktem apod., 
%   je zapotřebí u třídy použít parametry english a enslovak následovně:
%   If the work is written in English with the Slovak abstract, etc., 
%   it is necessary to use parameters english and enslovak as follows:
%      \documentclass[english,enslovak]{fitthesis}

% Základní balíčky jsou dole v souboru šablony fitthesis.cls
% Basic packages are at the bottom of template file fitthesis.cls
% zde můžeme vložit vlastní balíčky / you can place own packages here

% Kompilace po částech (rychlejší, ale v náhledu nemusí být vše aktuální)
% Compilation piecewise (faster, but not all parts in preview will be up-to-date)
% \usepackage{subfiles}

% Nastavení cesty k obrázkům
% Setting of a path to the pictures
%\graphicspath{{obrazky-figures/}{./obrazky-figures/}}
%\graphicspath{{obrazky-figures/}{../obrazky-figures/}}

%---rm---------------
\renewcommand{\rmdefault}{lmr}%zavede Latin Modern Roman jako rm / set Latin Modern Roman as rm
%---sf---------------
\renewcommand{\sfdefault}{qhv}%zavede TeX Gyre Heros jako sf
%---tt------------
\renewcommand{\ttdefault}{lmtt}% zavede Latin Modern tt jako tt

% vypne funkci šablony, která automaticky nahrazuje uvozovky,
% aby nebyly prováděny nevhodné náhrady v popisech API apod.
% disables function of the template which replaces quotation marks
% to avoid unnecessary replacements in the API descriptions etc.
\csdoublequotesoff



\usepackage{url}
\usepackage[english, czech]{babel}
\usepackage{blindtext}
\usepackage{subcaption}
\usepackage{graphicx}
\usepackage{listings}




% =======================================================================
% balíček "hyperref" vytváří klikací odkazy v pdf, pokud tedy použijeme pdflatex
% problém je, že balíček hyperref musí být uveden jako poslední, takže nemůže
% být v šabloně
% "hyperref" package create clickable links in pdf if you are using pdflatex.
% Problem is that this package have to be introduced as the last one so it 
% can not be placed in the template file.
\definecolor{codegreen}{rgb}{0,0.6,0}
\definecolor{codegray}{rgb}{0.5,0.5,0.5}
\definecolor{codepurple}{rgb}{0.58,0,0.82}
\definecolor{backcolour}{rgb}{0.95,0.95,0.92}
\definecolor{main-color}{rgb}{0.6627, 0.7176, 0.7764}
\definecolor{back-color}{rgb}{0.1686, 0.1686, 0.1686}
\definecolor{string-color}{rgb}{0.3333, 0.5254, 0.345}
\definecolor{key-color}{rgb}{0.8, 0.47, 0.196}

\lstdefinestyle{mystyle}{
	otherkeywords={Vector3, var, foreach, ;, TileBase, string, Vector3Int, public, override},
	backgroundcolor = {\color{back-color}},
	commentstyle=\color{codegreen},
	keywordstyle = {\color{key-color}},
	numberstyle=\tiny\color{codegray},
	stringstyle = {\color{string-color}},
	basicstyle = {\ttfamily \color{main-color}},
	breakatwhitespace=false,         
	breaklines=true,                 
	captionpos=b,                    
	keepspaces=true,                 
	numbers=left,                    
	numbersep=5pt,                  
	showspaces=false,                
	showstringspaces=false,
	showtabs=false,                  
	tabsize=2,
	language = C++,
}


\lstset{style=mystyle}
\ifWis
\ifx\pdfoutput\undefined % nejedeme pod pdflatexem / we are not using pdflatex
\else
  \usepackage{color}
  \usepackage[unicode,colorlinks,hyperindex,plainpages=false,pdftex]{hyperref}
  \definecolor{hrcolor-ref}{RGB}{223,52,30}
  \definecolor{hrcolor-cite}{HTML}{2F8F00}
  \definecolor{hrcolor-urls}{HTML}{092EAB}
  \hypersetup{
	linkcolor=hrcolor-ref,
	citecolor=hrcolor-cite,
	filecolor=magenta,
	urlcolor=hrcolor-urls
  }
  \def\pdfBorderAttrs{/Border [0 0 0] }  % bez okrajů kolem odkazů / without margins around links
  \pdfcompresslevel=9
\fi
\else % pro tisk budou odkazy, na které se dá klikat, černé / for the print clickable links will be black
\ifx\pdfoutput\undefined % nejedeme pod pdflatexem / we are not using pdflatex
\else
  \usepackage{color}
  \usepackage[unicode,colorlinks,hyperindex,plainpages=false,pdftex,urlcolor=black,linkcolor=black,citecolor=black]{hyperref}
  \definecolor{links}{rgb}{0,0,0}
  \definecolor{anchors}{rgb}{0,0,0}
  \def\AnchorColor{anchors}
  \def\LinkColor{links}
  \def\pdfBorderAttrs{/Border [0 0 0] } % bez okrajů kolem odkazů / without margins around links
  \pdfcompresslevel=9
\fi
\fi
% Řešení problému, kdy klikací odkazy na obrázky vedou za obrázek
% This solves the problems with links which leads after the picture
\usepackage[all]{hypcap}

% Informace o práci/projektu / Information about the thesis
%---------------------------------------------------------------------------
\projectinfo{
  %Prace / Thesis
  project={BP},            %typ práce BP/SP/DP/DR  / thesis type (SP = term project)
  year={2021},             % rok odevzdání / year of submission
  date=\today,             % datum odevzdání / submission date
  %Nazev prace / thesis title
  title.cs={Generování 2D map pro počítačové hry},  % název práce v češtině či slovenštině (dle zadání) / thesis title in czech language (according to assignment)
  title.en={2D map generation for computer games}, % název práce v angličtině / thesis title in english
  %title.length={14.5cm}, % nastavení délky bloku s titulkem pro úpravu zalomení řádku (lze definovat zde nebo níže) / setting the length of a block with a thesis title for adjusting a line break (can be defined here or below)
  %sectitle.length={14.5cm}, % nastavení délky bloku s druhým titulkem pro úpravu zalomení řádku (lze definovat zde nebo níže) / setting the length of a block with a second thesis title for adjusting a line break (can be defined here or below)
  %dectitle.length={14.5cm}, % nastavení délky bloku s titulkem nad prohlášením pro úpravu zalomení řádku (lze definovat zde nebo níže) / setting the length of a block with a thesis title above declaration for adjusting a line break (can be defined here or below)
  %Autor / Author
  author.name={Kryštof},   % jméno autora / author name
  author.surname={Glos},   % příjmení autora / author surname 
  %author.title.p={Bc.}, % titul před jménem (nepovinné) / title before the name (optional)
  %author.title.a={Ph.D.}, % titul za jménem (nepovinné) / title after the name (optional)
  %Ustav / Department
  department={UPGM}, % doplňte příslušnou zkratku dle ústavu na zadání: UPSY/UIFS/UITS/UPGM / fill in appropriate abbreviation of the department according to assignment: UPSY/UIFS/UITS/UPGM
  % Školitel / supervisor
  supervisor.name={Michal},   % jméno školitele / supervisor name    % příjmení školitele / supervisor surname
  supervisor.title.p={Ing.},   %titul před jménem (nepovinné) / title before the name (optional)
  supervisor.surname={Matýšek},    %titul za jménem (nepovinné) / title after the name (optional)
  % Klíčová slova / keywords
  keywords.cs={Procedurální generování obsahu, 2D mapy, 2D hry, počítačové hry, hry o přežití}, % klíčová slova v českém či slovenském jazyce / keywords in czech or slovak language
  keywords.en={Procedural content generation, 2D maps, 2D games, computer games, survival games}, % klíčová slova v anglickém jazyce / keywords in english
  %keywords.en={Here, individual keywords separated by commas will be written in English.},
  % Abstrakt / Abstract
  abstract.cs={Tato práce se zaměřuje na jednotlivé metody procedurálního generování z hlediska herního průmyslu, zejména v oblasti vytváření map. Cílem této práce, je vytvoření hry žánru Colony-sim, využívající procedurálního generování obsahu. 
  	
  	Zaměřil jsem se na analýzu různých způsobů vytváření obsahu a rozbor metod pomocí kterých lze automatizovaně generovat herní prostředí. 
  	
  	Zvolenou metodou pro tvoření herní oblasti byl Perlinův šum, s následným využitím výškových map. V práci představuji algoritmus, jenž na základě vstupních parametrů generuje funkční, unikátní a neopakující se mapy, což zajišťuje opakovatelnou hratelnost. Klíčovým výstupem této práce, je hra implementující procedurální algoritmus generující mapy, který je aplikovatelný v širokém spektru 2D her.}, % abstrakt v českém či slovenském jazyce / abstract in czech or slovak language
  abstract.en={This work focuses on individual methods of procedural generation in the context of the gaming industry, particularly in the realm of map creation. The aim of this study is to develop a Colony-sim genre game utilizing procedural content generation.
  	
  	I have concentrated on analyzing various methods of content creation and scrutinizing techniques through which gaming environments can be automatedly generated.
  	
  	The chosen method for crafting the gaming area was Perlin noise, coupled with the subsequent utilization of height maps. In this work, I present an algorithm that, based on input parameters, generates functional, unique, and non-repeating maps, ensuring repetitive playability. The key outcome of this study is the implementation of a game incorporating a procedural algorithm for map generation, applicable across a broad spectrum of 2D games.}, % abstrakt v anglickém jazyce / abstract in english
  %abstract.en={An abstract of the work in English will be written in this paragraph.},
  % Prohlášení (u anglicky psané práce anglicky, u slovensky psané práce slovensky) / Declaration (for thesis in english should be in english)
  declaration={Prohlašuji, že jsem tuto bakalářskou práci vypracoval samostatně pod vedením pana X...
Další informace mi poskytli...
Uvedl jsem všechny literární prameny, publikace a další zdroje, ze kterých jsem čerpal.},
  %declaration={I hereby declare that this Bachelor's thesis was prepared as an original work by the author under the supervision of Mr. X
% The supplementary information was provided by Mr. Y
% I have listed all the literary sources, publications and other sources, which were used during the preparation of this thesis.},
  % Poděkování (nepovinné, nejlépe v jazyce práce) / Acknowledgement (optional, ideally in the language of the thesis)
  acknowledgment={V této sekci je možno uvést poděkování vedoucímu práce a těm, kteří poskytli odbornou pomoc
(externí zadavatel, konzultant apod.).},
  %acknowledgment={Here it is possible to express thanks to the supervisor and to the people which provided professional help
%(external submitter, consultant, etc.).},
  % Rozšířený abstrakt (cca 3 normostrany) - lze definovat zde nebo níže / Extended abstract (approximately 3 standard pages) - can be defined here or below
  %extendedabstract={Do tohoto odstavce bude zapsán rozšířený výtah (abstrakt) práce v českém (slovenském) jazyce.},
  %extabstract.odd={true}, % Začít rozšířený abstrakt na liché stránce? / Should extended abstract start on the odd page?
  %faculty={FIT}, % FIT/FEKT/FSI/FA/FCH/FP/FAST/FAVU/USI/DEF
  faculty.cs={Fakulta informačních technologií}, % Fakulta v češtině - pro využití této položky výše zvolte fakultu DEF / Faculty in Czech - for use of this entry select DEF above
  faculty.en={Faculty of Information Technology}, % Fakulta v angličtině - pro využití této položky výše zvolte fakultu DEF / Faculty in English - for use of this entry select DEF above
  department.cs={Ústav matematiky}, % Ústav v češtině - pro využití této položky výše zvolte ústav DEF nebo jej zakomentujte / Department in Czech - for use of this entry select DEF above or comment it out
  department.en={Institute of Mathematics} % Ústav v angličtině - pro využití této položky výše zvolte ústav DEF nebo jej zakomentujte / Department in English - for use of this entry select DEF above or comment it out
}

% Rozšířený abstrakt (cca 3 normostrany) - lze definovat zde nebo výše / Extended abstract (approximately 3 standard pages) - can be defined here or above
%\extendedabstract{Do tohoto odstavce bude zapsán výtah (abstrakt) práce v českém (slovenském) jazyce.}
% Začít rozšířený abstrakt na liché stránce? / Should extended abstract start on the odd page?
%\extabstractodd{true}

% nastavení délky bloku s titulkem pro úpravu zalomení řádku - lze definovat zde nebo výše / setting the length of a block with a thesis title for adjusting a line break - can be defined here or above
%\titlelength{14.5cm}
% nastavení délky bloku s druhým titulkem pro úpravu zalomení řádku - lze definovat zde nebo výše / setting the length of a block with a second thesis title for adjusting a line break - can be defined here or above
%\sectitlelength{14.5cm}
% nastavení délky bloku s titulkem nad prohlášením pro úpravu zalomení řádku - lze definovat zde nebo výše / setting the length of a block with a thesis title above declaration for adjusting a line break - can be defined here or above
%\dectitlelength{14.5cm}

% řeší první/poslední řádek odstavce na předchozí/následující stránce
% solves first/last row of the paragraph on the previous/next page
\clubpenalty=10000
\widowpenalty=10000

% checklist
\newlist{checklist}{itemize}{1}
\setlist[checklist]{label=$\square$}

% Nechcete-li, aby se u oboustranného tisku roztahovaly mezery pro zaplnění stránky, odkomentujte následující řádek / If you do not want enlarged spacing for filling of the pages in case of duplex printing, uncomment the following line
% \raggedbottom

\begin{document}
  % Vysazeni titulnich stran / Typesetting of the title pages
  % ----------------------------------------------
  \maketitle
  % Obsah
  % ----------------------------------------------
  \setlength{\parskip}{0pt}

  {\hypersetup{hidelinks}\tableofcontents}
  
  % Seznam obrazku a tabulek (pokud prace obsahuje velke mnozstvi obrazku, tak se to hodi)
  % List of figures and list of tables (if the thesis contains a lot of pictures, it is good)
  \ifczech
    \renewcommand\listfigurename{Seznam obrázků}
  \fi
  \ifslovak
    \renewcommand\listfigurename{Zoznam obrázkov}
  \fi
  % {\hypersetup{hidelinks}\listoffigures}
  
  \ifczech
    \renewcommand\listtablename{Seznam tabulek}
  \fi
  \ifslovak
    \renewcommand\listtablename{Zoznam tabuliek}
  \fi
  % {\hypersetup{hidelinks}\listoftables}

  \ifODSAZ
    \setlength{\parskip}{0.5\bigskipamount}
  \else
    \setlength{\parskip}{0pt}
  \fi

  % vynechani stranky v oboustrannem rezimu
  % Skip the page in the two-sided mode
  \iftwoside
    \cleardoublepage
  \fi

  % Text prace / Thesis text
  % ----------------------------------------------
  \ifenglish
    \input{projekt-01-kapitoly-chapters-en}
  \else
    % Autor: Kryštof Glos

\chapter*{Úvod}
Procedurálně vytvářený obsah v herním průmyslu je velmi důležitou součástí her už po několik let. Mnoho her postavených na této vlastnosti už se prosadilo na trhu a stále více se uplatňuje. Náhodné vytváření obsahu se používá například na tvoření herních map, věcí v místnosti, skládání různých dopředu vytvořených místností tak aby vznikla jedinečná mapa, zkrátka skoro všude. 

Hlavním důvodem používání náhodně vytvořeného obsahu je možnost opakovaného hraní stejné hry, bez toho aby se hra stala nezáživnou. V budoucnu se procedurální generování rozhodně neztratí a naopak se bude ještě více využívat v různých odvětvích. Cíl této práce je vytvořit jedinečnou 2D hru z pohledu z hora, která právě takové náhodné vytváření bude používat. Hra je o přežití a ovládání skupiny domorodců a nastolení míru na ostrově. V kapitole \ref{Teorie} je detailně popsáno náhodné generování a různé metody,v kapitole \ref{solution} je vysvětleno jaké způsoby budou nejlepší k řešení, jak byla práce naplánována a proč. V kapitole \ref{realization} je popsaný způsob jakým byla hra řešena, použité metody a jejich význam.

\chapter{Teorie} 
\label{Teorie}
Ve světě her jsou dvě možnosti jak přistupovat ke tvoření obsahu, jedním z těchto způsobů je tradiční nebo také mechanické. \hyperref[traditional]{Mechanické generování} je nejjednodušší, ale také nejpracnější. Další možností vytváření obsahu je pomocí metod implementující náhodné, nebo také \hyperref[procedural]{procedurální generování obsahu}. (anglicky Procedural Generation dále jen PG) je složitější na implementaci, avšak jakmile je implementované tak už žádné navrhování úrovní není třeba, takové úrovně však mohou mít chyby pokud nejsou správně ošetřeny. (například nedostupné místnosti, rostlina na vodě, atd.)

Hry které mají pouze dvě dimenze se nazývají 2D. (z anglického two dimensions) Je mnoho žánrů 2D her, RPG (role playing game) hry na hrdiny s příběhem, strategií, Co-op (kooperační) které jsou postavené na spolupráci více hráčů, survival (hry o přežití), colony-sim (z anglického colonization simulation) které mají simulovat kolonizaci ovládanou hráčem a tak dále. Tato bakalářská práce se zabývá hrou žánru colony-sim. Je mnoho způsobů jak vyvíjet takovou hru, nejčastěji se používají takzvané \hyperref[enginy]{herní enginy}, které takové vyvíjení hry ulehčují a jsou na to stavěné.


\section{Způsoby generování obsahu}
V této části porovnáme mechanické generování obsahu s procedurálním, vysvětlíme co je lepší kdy použít a jaké známe metody pro procedurální generování. Následující tři body popisují faktory, které je třeba zvážit při rozhodování, zda bude využita nějaká procedurální metoda na generování, nebo bude lepší použít manuální design úrovní a obsahu:
\begin{description}
	\item[Žánr tvořené hry] Při vytváření například takové FPS hry, u které záleží hlavně na ovládání a souboji hráče proti hráči, není třeba vytvářet mapy a další obsah procedurálně. Většinou stačí vytvořit například pouze jednu úroveň a na to není potřeba požívat procedurální generování. Při hrách které závisí na okolí, surovinách a přežití kde každá okolnost nějak ovlivňuje hráče, už procedurální generování hraje větší roli. Zároveň pro vytváření obsahu jako je text, může být PG velmi obtížnou volbou, neboť v RPG hrách bývá textová část velmi důležitá a generování textu je stále dost obtížné
	\item[Opakovaná hratelnost] Některé hry jsou dělané tak že čím déle hráč hraje stejnou úroveň(anglicky level) tím více se zlepšuje a je za to například odměňován je právě manuální tvoření úrovní lepší než procedurální. Naopak u jiných her, které mají úrovně procedurálně generované, většinou hráč danou úroveň zvládne, pokračuje na další a nepředpokládá se že se k ní bude ještě vracet.
	\item[Aspekt designu hry] Jestliže hra závisí z velké části na jedné úrovni s jejími mechanikami, vlastnostmi a obsahem, pak je lepší ji vytvářet mechanicky a doladit všechny vlastnosti a interakce s hráčem.
\end{description}


\subsection{Mechanické generování obsahu}
\label{traditional}
Mechanický typ generování je jedním z nejobvyklejších tvoření obsahu ve hrách. Používá se převážně v žánrech, jako je RPG (Role Play Game), RTS(Real Time Strategy) a další, ve kterých pozice objektů a struktura mapy hraje velkou roli a bez lidské tvorby by nebylo dosaženo potřebných výsledků. Toto tvoření lze interpretovat jako proces u něhož se návrhář za pomoci různých nástrojů, které postupně aplikuje, snaží dosáhnout požadovaného výsledku. Jde tedy o metodu ručního vytváření obsahu kde designér, nebo grafik navrhuje a postupně vytváří úroveň, či jinou část hry tak, aby vyhovovala potřebám, ať už se jedná o pozici stromu, nebo o to co hráči sděluji NPC.
	
Proces takového vytváření obsahu pro hry se dá zjednodušeně představit na příkladu. Uvažujme například truhláře vyrábějícího stůl, který také používá různé nástroje na zhotovení nábytku. Aplikováním těchto nástrojů na materiál postupně vytváří výsledný stůl. Nástroje a suroviny, které si na počátku celého procesu vybral, určí možné výsledky, takže výsledkem jeho práce nemůže být třeba obraz.

Díky tomuto přístupu nehrozí nesrovnalosti ve výsledku, například se nemůže stát že určitá část mapy bude nedostupná, nebo úplně nesmyslná. Další výhodou je, že výsledek bude přesně takový jak byl naplánovaný, avšak takovéto tvoření je u větších her, kde je nutná opětovná hratelnost, velmi časově náročné.

\subsection{Procedurální generování obsahu}
\label{procedural}
Roden and Parberry \cite{FromArtistry} pojmenovávají tento druh algoritmů \textit{amplifikační algoritmy (amplification algorithms)}, přijímají menší množství vstupních informací, které zpracují a vracejí větší objem dat na výstupu. Hendrikx et al. \cite{Hendrikx} pojímají procedurální generování jako alternativu k mechanickému navrhování obsahu, ale kladou důraz na zdokonalování a přidávání parametrů umožňujících zásah návrháře do takto vygenerovaných objektů.

Tento typ vytváření obsahu se používá ve více žánrech, ale asi nejznámější z hlediska generování map je Roguelike, kde každá nová hra má unikátní náhodnou mapu. PG se ovšem nepoužívá pouze na generování map, ale také na vytváření objektů, jako jsou stromy, textury, animace, text a další. Procedurální generování obsahu není to stejné jako, jak se mnozí domnívají, náhodné generování obsahu.
\begin{figure}[h]
	\centering
	\includegraphics[scale=0.33]{obrazky-figures/BindingOfIsaac.jpg}
	\caption{Příklad hry Roguelike žánru jménem Binding of Isaac, vpravo nahoře je vidět mapa dungeonu, která je procedurálně vygenerovaná, i takto může PG vypadat.}
\end{figure}


V zásadě je to proces obrácený jak u mechanického generování obsahu. Uživatel sice stále definuje různé nástroje, které jsou použity pro vytváření obsahu, ale nikoli pro své vlastní použití, ale naopak je vytváří pro algoritmus. Uživatel dále určuje pravidla podle kterých se generátor musí řídit tak, aby se dobral kýženého výsledku.

\subsubsection{Příklad procedurálního generování lesů}
\label{proceduralExample}
Pro lepší představu uvedu příklad. Program má funkci procedurálního generátoru lesů na mapě, cílem tohoto programu je náhodně naskládat stromy s rozumnými rozestupy od sebe. Pomocí mechanického generování obsahu by designér jednoduše přidával modely stromů do té doby, dokud by mapa neodpovídala představám návrháře a nesplňovala by potřeby herního designu. V případě procedurálního generování, je nutné definovat nástroje, což je v tomto příkladě schopnost přidávat stromy na mapu a pravidla, jako například kolik stromů je potřeba aby algoritmus vysázel, kam program může jednotlivé stromy pokládat a jaké jsou minimální rozestupy mezi nimi. Pravidla a nástroje nám tím pádem jednoznačně definují množinu potenciálních výsledku.

\includegraphics[scale=0.3]{obrazky-figures/keep-calm.png}
\includegraphics[scale=0.3]{obrazky-figures/keep-calm.png}

\todo{dva obrázky, jeden se stromy moc u sebe a druhý lepší při použití více pravidel}

\section{Možnosti využití v herním průmyslu}
\label{metody}
Algoritmů na generování obsahu existuje mnoho, každý používá jiné nástroje, ale všechny se musí podrobovat pravidlům která stanovuje programátor a podle kterých se řídí. Je více způsobů a míst kde se dá PG uplatnit, různé způsoby a důvody jsou popsány v této kapitole.

\subsection{Text}
Skoro všechny hry používají text. Z důvodu že každá informace v textu musí odpovídat realitě, je nutno velké množství omezení pro generování. Například když je v textu informace že král je mrtvý \cite{liuDeep}, musí být toto tvrzení pravdivé.

Velkým plusem procedurálního generování textu je vyprávění \cite{madoc59000}, takto vytvořené příběhy jsou často kreativnější a zajímavější, než ty co by vytvořil člověk, neboť lidé mají sklony psát příběhy které již slyšeli, nebo ze svých zkušeností, což dost omezuje nápady.

\subsection{Textury}
Textury jsou snad skoro ve všech 3D a velké části 2D her. Zároveň se jedná o obsah který má nejméně omezení pro generování. Jednou z nejčastějších metod, která se používá na tvoření textur, je \todo{odkaz na perlinův šum}Perlinův šum, který je detailněji popsán v kapitole \todo{tady a tady}.

\subsection{Zvuky a hudba}
Většina her má soundtrack a zvukové efekty. Soundtrack většinou nemá nijak zvlášť přísná pravidla, ale zvukové efekty musejí být výstižné a odpovídající akci v daný moment. 

Jukebox \cite{Dhariwal2020JukeboxAG} je model který dokáže generovat hudbu se zpěvem v originální nezpracované formě zvukových dat, s délkou v řádu minut, i s určením žánru a vokálního stylu. Modelů jako je tento již existuje více, avšak zatím to nejsou plně hodnotné soundtracky pro hry a ještě chvíli potrvá, než bude možné jednoduše vygenerovat hudbu a efekty pro hru pomocí pouhého nástroje.

\subsection{Krajina a úrovně}
Nejvíce obvyklý obsah který se ve hrách generuje a který je zároveň hlavním zaměřením této práce, jsou krajiny a úrovně. Generování lokací, úrovní, nebo obsahu mapy lze jak u 2D her, tak u 3D her. Za úroveň nebo oblast lze označovat otevřené, třeba krajina s lesy, i uzavřené prostranství, vnitřek budovy, nebo jeskyně. 

\paragraph*{Generování oblastí lze rozdělit následovně:}
\begin{description}
	\item[Vytváření půdorysu] Tímto pojmem se rozumí vytváření samotné podstaty oblasti. Zahrnuje vytvoření země, moří, řek, hor, atd. Všechny tyto oblasti jsou otevřeného typu. Také můžeme rozumět pod generováním půdorysu i vytváření uzavřeného typu oblastí, například rozložení jednotlivých místností jeskynního komplexu, vytvoření bludiště a jeho cest. 
	
	\item[Přidání vegetace a objektů] Tento bod v podstatě navazuje na předchozí, je potřeba abychom měli vytvořený půdorys, aby bylo kam přidávat a rozmísťovat další objekty. Jedná se o bod do kterého spadá vegetace, budovy, atd. 
	
	\begin{figure}
		\centering
		\begin{subfigure}[b]{0.475\textwidth}
			\includegraphics[scale=0.3]{obrazky-figures/keep-calm.png}
			\caption{Vygenerovaný půdorys oblasti}
		\end{subfigure}
		\begin{subfigure}[b]{0.475\textwidth}
			\includegraphics[scale=0.3]{obrazky-figures/keep-calm.png}
			\caption{Přidané stromy k půdorysu}
		\end{subfigure}
		\caption{Na obrázcích jsou vidět různé postupy generování obsahu, na obrázku a je vygenerovaný půdorys oblasti i s mořem a skálou, na obrázku b se k tomu přidaly stromy}
	\end{figure}

	
\end{description} 


\todo{informace o všemožných známých metodách procedurálního generování}

\includegraphics[scale=0.3]{obrazky-figures/keep-calm.png}

\textcolor{gray}{\blindtext[4]}


\subsubsection{Porovnání metod}
\todo{porovnání jednotlivých metod}
\textcolor{gray}{\blindtext[8]}

\includegraphics[scale=0.3]{obrazky-figures/keep-calm.png}

\textcolor{gray}{\blindtext[23]}


\section{2D hry}

\todo{popis hry}
\textcolor{gray}{\blindtext[18]}
\includegraphics[scale=0.3]{obrazky-figures/keep-calm.png}

\subsection{Enginy na vývoj her}
\label{enginy}
\includegraphics[scale=0.3]{obrazky-figures/keep-calm.png}

\textcolor{gray}{\blindtext[18]}

\chapter{Návrh řešení}
\label{solution}
\textcolor{gray}{\blindtext[2]}
\textcolor{gray}{\blindtext[46]}

\section{Vybraná metoda generování}
\todo{podrobnější popis metody, výhody, nevýhody}
\textcolor{gray}{\blindtext[46]}

\chapter{Realizace, experimenty a vyhodnocení}
\label{realization}
\textcolor{gray}{\blindtext[94]}

\chapter{Závěr}
\label{end}
\textcolor{gray}{\blindtext[4]}
  \fi
  
  % Kompilace po částech (viz výše, nutno odkomentovat)
  % Compilation piecewise (see above, it is necessary to uncomment it)
  %\subfile{projekt-01-uvod-introduction}
  % ...
  %\subfile{chapters/projekt-05-conclusion}


  % Pouzita literatura / Bibliography
  % ----------------------------------------------
\ifslovak
  \makeatletter
  \def\@openbib@code{\addcontentsline{toc}{chapter}{Literatúra}}
  \makeatother
  \bibliographystyle{bib-styles/Pysny/skplain}
\else
  \ifczech
    \makeatletter
    \def\@openbib@code{\addcontentsline{toc}{chapter}{Literatura}}
    \makeatother
    \bibliographystyle{bib-styles/Pysny/czplain}
  \else 
    \makeatletter
    \def\@openbib@code{\addcontentsline{toc}{chapter}{Bibliography}}
    \makeatother
    \bibliographystyle{bib-styles/Pysny/enplain}
  %  \bibliographystyle{alpha}
  \fi
\fi
  \begin{flushleft}
  \bibliography{projekt-20-literatura-bibliography}
  \end{flushleft}

  % vynechani stranky v oboustrannem rezimu
  % Skip the page in the two-sided mode
  \iftwoside
    \cleardoublepage
  \fi

  % Prilohy / Appendices
  % ---------------------------------------------
  \appendix
\ifczech
  \renewcommand{\appendixpagename}{Přílohy}
  \renewcommand{\appendixtocname}{Přílohy}
  \renewcommand{\appendixname}{Příloha}
\fi
\ifslovak
  \renewcommand{\appendixpagename}{Prílohy}
  \renewcommand{\appendixtocname}{Prílohy}
  \renewcommand{\appendixname}{Príloha}
\fi
%  \appendixpage

% vynechani stranky v oboustrannem rezimu
% Skip the page in the two-sided mode
%\iftwoside
%  \cleardoublepage
%\fi
  
\ifslovak
%  \section*{Zoznam príloh}
%  \addcontentsline{toc}{section}{Zoznam príloh}
\else
  \ifczech
%    \section*{Seznam příloh}
%    \addcontentsline{toc}{section}{Seznam příloh}
  \else
%    \section*{List of Appendices}
%    \addcontentsline{toc}{section}{List of Appendices}
  \fi
\fi
  \startcontents[chapters]
  \setlength{\parskip}{0pt} 
  % seznam příloh / list of appendices
  % \printcontents[chapters]{l}{0}{\setcounter{tocdepth}{2}}
  
  \ifODSAZ
    \setlength{\parskip}{0.5\bigskipamount}
  \else
    \setlength{\parskip}{0pt}
  \fi
  
  % vynechani stranky v oboustrannem rezimu
  \iftwoside
    \cleardoublepage
  \fi
  
  % Přílohy / Appendices
  \ifenglish
    \input{projekt-30-prilohy-appendices-en}
  \else
    \input{projekt-30-prilohy-appendices}
  \fi
  
  % Kompilace po částech (viz výše, nutno odkomentovat)
  % Compilation piecewise (see above, it is necessary to uncomment it)
  %\subfile{projekt-30-prilohy-appendices}
  
\end{document}
