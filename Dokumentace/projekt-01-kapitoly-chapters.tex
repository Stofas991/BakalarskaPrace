% Autor: Kryštof Glos

\chapter*{Úvod}
Procedurálně vytvářený obsah v herním průmyslu je velmi důležitou součástí her už po několik let. Mnoho her postavených na tomto principu, se již prosadilo na trhu a stále více se uplatňuje. Náhodné vytváření obsahu se používá například na tvoření herních map, věcí v místnosti, skládání různých dopředu vytvořených místností tak aby vznikla jedinečná mapa. Takto implementované hry mají výhodu v opakované hratelnosti a nepředvídatelnosti.

Cílem této práce je návrh a vytvoření prototypu 2D hry, v herním enginu Unity, založené na procedurálním generování herního obsahu. Samotný generační algoritmus pracuje s Perlinovým šumem a výškovými mapami, ty slouží k vytvoření elevace různých bodů na mapě.

Inspirací na téma hry je počítačová hra RimWorld, která se řadí do her typu Colony-sim. Jedná se o žánr, ve kterém hráč ovládá skupinu lidí, kolonii, kterou se snaží, pomocí dobrého vedení, dovést k nějakému danému cíli. V navržené hře je úkolem hráče dostat kolonisty do stavu prosperity a mimo stav nouze o potravu a zdroje. Pohyb kolonistů využívá takzvané NavMesh agenty, kteří spolupracují s vytvořenou navigační sítí (NavMesh).

Vytváření mapy je stěžejní částí celé hry, určuje zdroje pro hráče a tudíž částečně i obtížnost celé hry. Ve hře se objevují nepřátelé, kteří mají jednoduchou umělou inteligenci a jsou jednou z překážek, se kterou se hráč musí během hry vypořádávat. Implementace generačních algoritmů, AI nepřátel a dalšího je popsána v kapitole \ref{implementace}.

Hra je implementována v herním enginu Unity, které je v dnešní době, jedním z nejvíce používaných vývojových nástrojů pro vývoj her. Mezi další oblíbená vývojová prostředí se řadí například, GameMaker, Unreal Engine, Godot. O vývoji v jednotlivých enginech pojednává kapitola \ref{engines}.

Po dokončení implementační části bylo zahájeno experimentování s herními systémy a následovné uživatelské testování. Průběh a výsledky této části práce jsou dále popsány v kapitole \ref{experiments}.

\chapter{Vytváření obsahu v herním světě} 
\label{theory}
Ve světě her jsou dvě možnosti jak přistupovat ke tvoření obsahu, jedním z těchto způsobů je tradiční nebo také \hyperref[traditional]{mechanické}, je jednodušší, vzhledem k tomu že není potřeba žádný algoritmus, ale pracnější. Další možností vytváření obsahu je pomocí metod implementující náhodné nebo také \hyperref[procedural]{procedurální generování obsahu}. 

Hry které mají pouze dvě dimenze se nazývají 2D (z anglického two dimensions). Je mnoho žánrů 2D her, RPG (role playing game) hry na hrdiny s příběhem, strategií, Co-op (kooperační) které jsou postavené na spolupráci více hráčů, survival (hry o přežití), colony-sim (z anglického colonization simulation) které mají simulovat kolonizaci ovládanou hráčem a tak dále. Tato bakalářská práce se zabývá hrou žánru colony-sim. Je mnoho způsobů jak vyvíjet takovou hru, nejčastěji se používají takzvané \hyperref[engines]{herní enginy}, které takové vyvíjení hry ulehčují a jsou na to stavěné.


\section{Způsoby generování obsahu}
V této části porovnáme mechanické generování obsahu s procedurálním, vysvětlíme co je lepší kdy použít a jaké známe metody pro procedurální generování. Následující tři body popisují faktory, které je třeba zvážit při rozhodování, zda bude využita nějaká procedurální metoda generování, nebo bude lepší použít manuální design úrovní a obsahu:
\begin{description}
	\item[Žánr tvořené hry] Při vývoji například takové hry z prvního pohledu (FPS) hry, u které záleží hlavně na ovládání a souboji hráče proti hráči, není třeba vytvářet mapy a další obsah procedurálně. Většinou stačí vytvořit například pouze jednu úroveň a v takovém případě není třeba používat procedurální generování. Při hrách, které závisí na okolí, surovinách a přežití kde každá okolnost nějak ovlivňuje hráče, už procedurální generování hraje větší roli. Zároveň pro vytváření obsahu jako je text, může být PG velmi obtížnou volbou, neboť v RPG hrách bývá textová část velmi důležitá a generování textu je stále dost obtížné
	\item[Opakovaná hratelnost] Některé hry jsou dělané tak, že čím déle hráč hraje stejnou úroveň (anglicky level), tím více se zlepšuje a je za to například odměňován je právě manuální tvoření úrovní lepší než procedurální. Naopak u jiných her, které mají úrovně procedurálně generované, většinou hráč danou úroveň zvládne, pokračuje na další a nepředpokládá se že se k ní bude ještě vracet.
	\item[Aspekt designu hry] Jestliže hra závisí z velké části na jedné úrovni s jejími mechanikami, vlastnostmi a obsahem, pak je lepší ji vytvářet mechanicky a doladit všechny vlastnosti a interakce s hráčem.
\end{description}


\subsection{Mechanické generování obsahu}
\label{traditional}
Mechanický typ generování je jedním z nejobvyklejších tvoření obsahu ve hrách. Používá se převážně v žánrech, jako je RPG (Role Play Game), RTS (Real Time Strategy) a další, ve kterých pozice objektů a struktura mapy hraje velkou roli a bez lidské tvorby by nebylo dosaženo potřebných výsledků. Toto tvoření lze interpretovat jako proces u něhož se návrhář za pomoci různých nástrojů, které postupně aplikuje, snaží dosáhnout požadovaného výsledku. Jde tedy o metodu ručního vytváření obsahu kde designér, nebo grafik navrhuje a postupně vytváří úroveň, či jinou část hry tak, aby vyhovovala potřebám, ať už se jedná o pozici stromu, nebo o to co hráči sděluji NPC.

Díky tomuto přístupu nehrozí nesrovnalosti ve výsledku, například se nemůže stát že určitá část mapy bude nedostupná, nebo úplně nesmyslná. Další výhodou je, že výsledek bude přesně takový jak byl naplánovaný, avšak takovéto tvoření je u větších her, kde je nutná opětovná hratelnost, velmi časově náročné.

\subsection{Procedurální generování obsahu}
Tento typ vytváření obsahu se používá ve více žánrech, ale asi nejznámější z hlediska generování map je Roguelike, kde každá nová hra má unikátní náhodnou mapu. Procedurální generování se ovšem nepoužívá pouze na generování map, ale také na vytváření objektů, jako jsou stromy, textury, animace, text a další. Procedurální generování obsahu není úplně to stejné, jako náhodné generování obsahu. Více do hloubky je tento typ modelování krajiny a textur rozebrán v kapitole \hyperref[procedural]{Procedurální generování}.

V zásadě je to proces obrácený jak u mechanického generování obsahu. Uživatel sice stále definuje různé nástroje, které jsou použity pro vytváření obsahu, ale nikoli pro své vlastní použití, ale naopak je vytváří pro algoritmus. Uživatel dále určuje pravidla podle kterých se generátor musí řídit tak, aby se dobral kýženého výsledku.

\subsubsection{Příklad procedurálního generování lesů}
\label{proceduralExample}
\todo{informace o tom kde konkrétně se používá + obrázky}
Pro lepší představu uvedu příklad. Program má funkci procedurálního generátoru lesů na mapě, cílem tohoto programu je náhodně naskládat stromy s rozumnými rozestupy od sebe. Pomocí mechanického generování obsahu by designér jednoduše přidával modely stromů do té doby, dokud by mapa neodpovídala představám návrháře a nesplňovala by potřeby herního designu. V případě procedurálního generování, je nutné definovat nástroje, což je v tomto příkladě schopnost přidávat stromy na mapu a pravidla, jako například kolik stromů je potřeba aby algoritmus vysázel, kam program může jednotlivé stromy pokládat a jaké jsou minimální rozestupy mezi nimi. Pravidla a nástroje nám tím pádem jednoznačně definují množinu potenciálních výsledku.

\begin{figure}[h]
	\centering
	\includegraphics[scale=0.33]{obrazky-figures/BindingOfIsaac.jpg}
	\caption{Příklad hry Roguelike žánru jménem Binding of Isaac, vpravo nahoře je vidět mapa dungeonu, která je procedurálně vygenerovaná.}
\end{figure}

\includegraphics[scale=0.3]{obrazky-figures/keep-calm.png}
\includegraphics[scale=0.3]{obrazky-figures/keep-calm.png}

\todo{dva obrázky, jeden se stromy moc u sebe a druhý lepší při použití více pravidel}

\section{Procedurální generování v herním průmyslu}
\label{proceduralInGames}
Algoritmů na generování obsahu existuje mnoho, každý používá jiné nástroje, ale všechny se musí podrobovat pravidlům která stanovuje programátor a podle kterých se řídí. Je více způsobů a míst kde se dá procedurální generování uplatnit, různé způsoby a důvody jsou popsány v této kapitole.

\subsection{Text}
Skoro všechny hry používají text. Z důvodu že každá informace v textu musí odpovídat realitě, je nutno velké množství omezení pro generování. Například když je v textu informace že král je mrtvý \cite{liuDeep}, musí být toto tvrzení pravdivé.

Velkým plusem procedurálního generování textu je vyprávění \cite{madoc59000}, takto vytvořené příběhy jsou často kreativnější a zajímavější, než ty co by vytvořil člověk, neboť lidé mají sklony psát příběhy které již slyšeli, nebo ze svých zkušeností, což dost omezuje nápady.

\subsection{Krajina a úrovně}
Nejvíce obvyklý obsah který se ve hrách generuje, a který je zároveň hlavním zaměřením této práce, jsou krajiny a úrovně. Generování lokací, úrovní, nebo obsahu mapy lze jak u 2D her, tak u 3D her. Za úroveň nebo oblast lze označovat otevřené, třeba krajina s lesy, i uzavřené prostranství, vnitřek budovy, nebo jeskyně. Tato část PG je rozvedená a podrobněji popsaná v následující kapitole \ref{Krajina}.

\subsection{Textury}
Zároveň se jedná o obsah který má nejméně omezení pro generování. Jednou z nejčastějších metod, která se používá na tvoření textur, je \todo{odkaz na perlinův šum} Perlinův šum, který je detailněji popsán v kapitole \todo{tady a tady}.

\subsection{Zvuky a hudba}
Většina her má soundtrack a zvukové efekty. Soundtrack obvykle nemá nijak zvlášť přísná pravidla, ale zvukové efekty musejí být výstižné a odpovídající akci v daný moment. 

Jukebox \cite{Dhariwal2020JukeboxAG} je model který dokáže generovat hudbu se zpěvem v originální nezpracované formě zvukových dat, s délkou v řádu minut, i s určením žánru a vokálního stylu. Modelů jako je tento již existuje více, avšak zatím to nejsou plně hodnotné soundtracky pro hry a ještě chvíli potrvá, než bude možné jednoduše vygenerovat hudbu a efekty pro hru pomocí pouhého nástroje.

\pagebreak

\chapter{Enginy na vývoj her}
\label{engines}
Herní engine představuje platformu složenou z interagujícího softwaru, který dohromady vytváří integrovaný celek a umožňuje spouštění samotných her. Herní engine se skládá z několika částí s přesně specifikovanou funkcionalitou: rendering, fyzika, síťování, zvuk atd. \cite{nilson2007game} 

Platforem na vývoj her existuje mnoho, některými z nich jsou například Unity, Construct 2, MonoGame, Unreal Engine, nebo GameMaker Studio 2. Každý herní engine je v něčem jiný a tudíž se hodí na jiné žánry, nebo styly her. Při rozhodování, který engine použít, se z hlediska vývojáře musí zohlednit vícero faktorů, například podporovaný programovací jazyk, nebo platformu na kterou je hra vyvíjena. \cite{vohera2021game}
\section{Unity}
\label{unity}
\includegraphics[scale=0.3]{obrazky-figures/keep-calm.png}

\textcolor{gray}{\blindtext[10]}

\chapter{Procedurální generování}
\label{procedural}
Tato kapitola popisuje \todo{konkrétní metody které popisuje} a použití procedurálního generování v herním průmyslu, do detailu popisuje PG krajiny, generování fraktálů.

Procedurální modelování je téma které se aktivně zkoumá už přes čtyřicet let. Myšlenka je, jak již bylo zmíněno, aby obsah který se vytvářel ručně, dal modelovat pomocí navržené procedury automaticky. Takovýto přístup se již uplatnil na generování například textur, geometrických modelů, zvukových nahrávek, nebo animací. V roce 1980 se začalo pracovat s různými metodami na vytváření terénu, jako hory, pláně a jezera. Začal se také řešit růst rostlin a obecně práce s přírodou. \cite{inproceedings}

Roden and Parberry \cite{FromArtistry} pojmenovávají tento druh algoritmů \textit{amplifikační algoritmy (amplification algorithms)}, přijímají menší množství vstupních informací, které zpracují a vracejí větší objem dat na výstupu. Hendrikx et al. \cite{Hendrikx} pojímají procedurální generování jako alternativu k mechanickému navrhování obsahu, ale kladou důraz na zdokonalování a přidávání parametrů umožňujících zásah návrháře do takto vygenerovaných objektů.

\section{Celulární automaty pro PG}
\label{celular}
Celulární automaty se ve hrách používají intenzivně zejména pro modelování týkajícího se systémů v prostředí, jako jsou teplo, oheň, déšť, tlak a exploze. Zatím podle průzkumu nejsou známé žádné hry, které by postavili generování celého 2D herního světa, pouze pomocí celulárních automatů. Momentálně existují webové stránky, které možnost generování malých map pomocí mřížek navrhují, ale není jich mnoho a neexistuje žádné spolehlivé ohodnocení těchto algoritmů. \cite{articleCellular}

Původně byly CA vymyšleny Johnem Von Neumannem jako formální model sebereprodukujících se organismů. Šlo o dvou-dimenzionální celulární automat, kde každá buňka, tzv. cell, je malý čtverec na velkém čtverečkovaném papíru. Každá buňka má dva možné stavy, černý a červený, které jsou určeny jejich sousedstvím. V John Von Neumannově teorii, je sousedství tvořené čtyřmi přilehlými čtverci a na obrázku \ref{vonNeumann} jsou vyznačeny červenou barvou. \cite{Gong2017}

Nejznámější celulární automat byl vytvořen v roce 1970 britským matematikem Johnem Hortonem Conwayem, který se jmenoval Game Of Life. Stejně jako Von Neumannův byl i tento automat dvou-dimenzionální a mohli nabývat pouze hodnot živá, nebo mrtvá. Využívá Moorovo sousedství, které oproti Von Neumannově považuje za sousední buňky všech osm přilehlých, vyobrazené na obrázku \ref{moore}. Fungování automatu je následovné, buňka zůstává naživu, pokud má dvě, nebo tři sousedící buňky živé. Což simuluje, že buňka nepřežije pokud je osamělá, ale zároveň pokud je okolí přeplněné organismy, tak je utlačována. Další pravidlo je, že pokud je libovolná buňka mrtvá, může se "narodit", pokud jsou v sousedství alespoň tři živé buňky. Toto pravidlo má simulovat rození, kde každá buňka musí mít tři rodiče. Automat díky těmto jednoduchým pravidlům dokáže vytvářet simulace které působily jako živý organismus. \cite{Gong2017}

\begin{figure}[h]
	\centering
	\begin{subfigure}{0.475\textwidth}
		\label{vonNeumann}
		\centering
		\includegraphics[scale=0.1]{obrazky-figures/Von_neumann_neighborhood.svg}
		\caption{Von Neumannovo sousedství}
	\end{subfigure}
	\begin{subfigure}{0.475\textwidth}
		\label{moore}
		\centering
		\includegraphics[scale=0.1]{obrazky-figures/Moore_neighborhood.svg}
		\caption{Moorovo sousedství}
	\end{subfigure}
	\caption{Sousedství z pohledu Von Neumanna a Moora}
\end{figure}

\section{Výškové mapy}
\label{heightMaps}
Výškové mapy představují dvoudimenzionální mřížky obsahující výškové hodnoty, jež jsou častým prvkem v modelování terénu. Tyto mapy se běžně využívají jako klíčový prvek pro reprezentaci základu terénu v herním průmyslu. Pro tvorbu výškových map existuje mnoho algoritmů.

Jedny z nejstarších algoritmů jsou metody založené na pododdělení. Segmenty v rámci vygenerované hrubé výškové mapy jsou iterativně rozdělovány, kde každá iterace navíc používá kontrolovanou náhodnost k přidávání detailů. Miller \cite{MillerRendering} popisuje některé varianty všeobecně známé metody středového posunu\todo{napsat něco o této metodě?}, ve které se výška nového bodu nastavuje na průměr hodnot jeho rohů v trojúhelníkovém nebo diamantovém tvaru, k němuž je přidán náhodný offset. Každou iterací se rozsah náhodných hodnot offsetu snižuje, podle parametru, který kontroluje hrubost výsledné výškové mapy. 

Generování výškových map se v dnešní době provádí převážně pomocí fraktálních generátorů šumu, jako je \todo{odkaz na sum}Perlinův šum, který provádí generování šumu vzorkováním a interpolací bodů na mřížce náhodných vektorů.

Výškové mapy se mohou dále transformovat na základě běžných filtrací obrazu, například vyhlazování(smoothing), nebo simulací fyzikálních jevů, např. eroze. \cite{inproceedings}

\section{Procedurální generování krajiny}
\label{terrain}
Procedurální generování krajiny se řadí ke složitějším tématům PG. Je tomu tak hlavně kvůli tomu, že se k tomuto typu generování většinou používají \hyperref[heightMaps]{výškové mapy} (height maps), které se skládají z bitových map odstínů šedé, ve kterých se výška udává pomocí právě odstínu šedé barvy daného pixelu. Většinou se používá postup, při kterém platí, že čím světlejší je pixel, tím větší je výška. Jakmile jsou pixely ohodnoceny, jsou tři různé způsoby jak postupovat:
\begin{itemize}
	\item ruční vykreslování textur,
	\item aplikování textur na ručně vytvořené regiony podle výšky,
	\item vygenerování textur po analyzování výšek na vygenerované výškové mapě.
\end{itemize}
První metoda je výhodná v tom, že textury ručně vykreslené lze udělat na míru a žádný algoritmus je nemůže replikovat. Bohužel jsou časově náročné a jejich kvalita přímo závisí na schopnostech výtvarníka.

Druhá metoda je podobná, nakreslí se barvy na určitá místa, kam si designer myslí že by se mohly hodit hory, řeky, nebo země. Jedná se o docela obvyklou metodu, která přináší velice kvalitní výsledky, ale stále se jedná o manuální techniku.

Třetí metoda používá data height mapy a vypočítává jakou texturu použít na které místo. Nízké hodnoty (tmavší místa) se používají jako oceány, střední hodnoty se vyhodnocují jako země/tráva a na vysoké hodnoty (světlá místa) se nanášejí textury hor nebo kamení. Při generování ve 3D hrách lze ještě započítávat takzvané sklony (slopes), díky kterým je možné oddělovat vyšší plochy od nižších pomocí dalších textur například kamene.

Vzhledem k tomu že první dva postupy vyžadují mechanické vykreslování buď barvy, nebo textur on designéra, nedá se o nich mluvit jako o čistě procedurálních metodách. Naopak postup třetí tomuto popisu zcela odpovídá, neboť kompletně závisí na height mapě a není nutná žádná akce návrháře \cite{madoc59000}.

\pagebreak
\paragraph*{Generování oblastí lze rozdělit následovně:}
\begin{description}
	\item[Vytváření půdorysu] Zahrnuje vytvoření země, moří, řek, hor, atd. Všechny tyto oblasti jsou otevřeného typu. Také můžeme rozumět pod generováním půdorysu i vytváření uzavřeného typu oblastí, například rozložení jednotlivých místností jeskynního komplexu, vytvoření bludiště a jeho cest. 
	
	\item[Přidání vegetace a objektů] Tento bod v podstatě navazuje na předchozí, je potřeba abychom měli vytvořený půdorys, aby bylo kam přidávat a rozmísťovat další objekty. Jedná se o bod do kterého spadá vegetace, budovy, atd. Programu se zadávají různá omezení na množství vegetace, místa kam který objekt lze přidávat a další omezení.
\end{description} 

\begin{figure}[h]
	\centering
	\begin{subfigure}{0.475\textwidth}
		\centering
		\includegraphics[scale=0.3]{obrazky-figures/keep-calm.png}
		\caption{Vygenerovaný půdorys oblasti}
	\end{subfigure}
	\begin{subfigure}{0.475\textwidth}
		\centering
		\includegraphics[scale=0.3]{obrazky-figures/keep-calm.png}
		\caption{Přidané stromy k půdorysu}
	\end{subfigure}
	\caption{Na obrázcích jsou vidět různé postupy generování obsahu, na obrázku a je vygenerovaný půdorys oblasti i s mořem a skálou, na obrázku b se k tomu přidaly stromy}
\end{figure}

\todo{informace o všemožných známých metodách procedurálního generování}

\includegraphics[scale=0.3]{obrazky-figures/keep-calm.png}

\section{Metody pro procedurální generování krajiny}
\subsubsection{Porovnání metod}
\todo{porovnání jednotlivých metod}
\textcolor{gray}{\blindtext[8]}

\includegraphics[scale=0.3]{obrazky-figures/keep-calm.png}

\textcolor{gray}{\blindtext[23]}

\section{2D hry}

\todo{popis hry}
\textcolor{gray}{\blindtext[18]}
\includegraphics[scale=0.3]{obrazky-figures/keep-calm.png}

\chapter{Návrh řešení}
\label{solution}
\textcolor{gray}{\blindtext[2]}
\textcolor{gray}{\blindtext[46]}

\section{Vybraná metoda generování}
\todo{podrobnější popis metody, výhody, nevýhody}
\textcolor{gray}{\blindtext[46]}

\chapter{Implementace}
\label{implementace}
Tato část se věnuje podrobnostem implementace skriptů, jež jsou klíčovými součástmi vytváření finální hry.
\textcolor{gray}{\blindtext[60]}
\chapter{Experimenty a vyhodnocení}
\label{experiments}
\section{testování}
\label{tests}
\textcolor{gray}{\blindtext[30]}

\chapter{Závěr}
\label{end}
\textcolor{gray}{\blindtext[4]}